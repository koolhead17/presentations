% India's e-governance policy 
\documentclass{beamer}
\usetheme{CambridgeUS}
\author{Atul Kumar Jha}
\title{FLOSS adoption by the govt. of India}
\institute{CSS Corp. India }

\begin{document}

	\begin{frame}{}
	\titlepage
	\end{frame}
        \begin{frame}{Then...}
                \begin{itemize}[<+->]
                          \item IT was new and adoption was challenge.
                          \item There was no governing body to encourage/promote FOSS.
                          \item E-governance was still in it's infancy.
                                \begin{itemize}
                                    \item No government policy/vision in promoting E-Governence.
                                     \item Earliest adoptions were on non-free platform.
                                 \end{itemize}
                          \item ICT was not in picture.
                \end{itemize}
        \end{frame}
        \begin{frame}{Now...}
                  \begin{itemize}[<+->]
                      \item TODO: add FOSS webpage of deptt of IT. %removed
          %photo needs to be added here
                      \item There is a change in Govenment`s approach in implementing Open Source.
                            \begin{itemize}
                                \item National Resource Centre for Free/Open Source Software (NRCFOSS) % modified
                                \item Open technology centre launched.%modified
                             \end{itemize}
                      \end{itemize}
        \end{frame} 
        \begin{frame}{Flow of Presentation}
            \begin{itemize}[<+->]
                \item E-Governance initiative.
                       \begin{itemize}
                                \item NRCFOSS
                                \item Open Technology Centre (with NIC)
                                \item Assam state government
                       \end{itemize}
                \item ICT initiative.
                         \begin{itemize}
                                  \item National Mission on Education through ICT (NME-ICT).
                                  \item Kerala state govenment ICT initiative. 
                         \end{itemize}
                 
                  \item Others
                          \begin {itemize}
                                \item Open Source drug discovery project.
                          \end {itemize}
                   \item Conclusion
                   \item Videos
          \end {itemize}
      \end{frame}
   
     \begin{frame}{National Mission on Education through ICT}

            \begin{itemize}[<+->]

                   \item Launched by the Ministry of Human Resources Development (MHRD), Government of India
                   \item Objective: to raise the levels of education in India
                   \item Outlay billion USD in the current plan period, ending on 31 March 2012.
                   \item Likely to continue in the next plan period also.

            \end{itemize}
     \end{frame}         
   \begin {frame} {NMEICT: Three Thrust Areas}

       \begin{itemize}
          \item Bandwidth:
                \begin{itemize}
                          \item 1 GBPs bandwidth to every one of 500+universities and their affiliated colleges

                 \end{itemize}
          \item Low cost access device
%kapil sibble photo comes here
          \item Content generation (FOSS effort at IIT mumbai)
                    \begin{itemize}
                    \item Textbook Companion
                    \item Spoken Tutorial
                     \end{itemize}
       \end{itemize}
   \end{frame}

  \begin{frame}{Objectives of MHRD’s Tablet}

      \begin{itemize}[<+->]

        \item To make computing available for every students
        \item To be able to connect to Internet
        \item To listen to educational video
        \item To carry out simple computations
        \item To connect to the Cloud
        \item Priced at 50 USD
      \end{itemize}

  \end{frame}

   \begin{frame}{MHRD`s Tablet Softwares}

       \begin{itemize}[<+->]

          \item GNU/Linux, Android
          \item Useful softwares to students, e.g.,Scilab, Python, gEDA
          \item Will unleash FOSS opportunities:
                      \begin{itemize}
                              \item Training, GNU/Linux distribution for millions of students
                     \end{itemize}

    \end{itemize}
    \end{frame}                                

   \begin{frame}{Content generation (FOSS effort at IIT mumbai)}

        \begin{itemize}[<+->]

          \item Textbook Companion
           \item Spoken Tutorials

        \end{itemize}
   \end{frame}

   \begin{frame} {Textbook Companion}
         \begin{itemize}[<+->]

                \item It is a documentation project
                \item How it works: 
                        \begin{itemize}
                           \item Choose any standard textbook
                            \item Code worked out examples into Scilab/python others.
                             \item Get the correctness certified by the subject expert
                            \item Get honorarium 
                            \item Summer internship option.
                         \end{itemize}
                \item Licence:
                           \begin{itemize}
                                \item Textbook companion released under CC license.
                            \end{itemize}

          \end{itemize}
   \end{frame}

   \begin{frame}{Spoken tutorial}
       \begin{itemize}[<+->]
           \item screencast + spoken explanation 
            \item Dubbing in all the regionallanguages
            \item Ten minute duration
            \item Tools used: Recordmydesktop, Camstudio
            \item Cover videos on most free softwares
             \item Honorium to contributors
              \item Objective
                    \begin{itemize}
                           \item Create more than 100,000 such videos.
                           \item Distribute it among all the colleges/universities
                     \end{itemize} 
              \item Sucess Story : Many workshops have been conducted using Spoken-Tutorial screencast.
                       \begin{itemize}
                              \item Scilab
                              \item Python
                               \item Orca
                               \item PHP/MySQL

                        \end{itemize}
                        \item Licence:
                           \begin{itemize}
                                \item Spoken tutorial released under CC
                            \end{itemize}
                           

       \end{itemize}
    \end{frame}

    \begin{frame}{Kerala state govenment ICT initiative}
      \begin{itemize}
%brief info and objectives for the same needs to be added as section.
           \item Achievements
                \begin{itemize}
                     \item World’s largest Simultaneous deployment of FOSS(Free and Open Software Systems) initiatives in Education sector
                      \item The IT practical examination in free software for about 15 lakh students, first of this kind in the world
                       \item IT Mela 2008 conducted entirely on free software platform was introduced to encourage IT initiatives among students and teachers
                        \item Winner of National e-governance Award for the Best Project and also the winner of World is Open Award 2008
                 \end{itemize}
             
               
           \end{itemize}
     \end{frame}
%modified work starts from here
\begin{frame}{NRCFOSS}
        \begin{itemize}
          \item National Resource Center for FOSS
          \item Support and Funding from the Dept of Information and Technology
          \item Objectives
              \begin{itemize}
                  \item Bridge the digital divide
                   \item Bring ICT to the masses
                    \item Strengthen the Indian software industry with FOSS technologies
                     \item Encourage community participation                   
               \end{itemize}
           \item Achivenments
               \begin{itemize}
                     \item Over 2,00,000+ deployments of Gnu/Linux
                      \item Huge saving of revenue
                       \item More than 100,000 students got introduced to FOSS
                        \item FOSS Support centre established in many parts of country 
                \end{itemize}
                   
        \end{itemize}
\end{frame}

\begin{frame}{Open Technology Centre}
    \begin{itemize}
      \item Set up in March 2007 in Chennai
       \item A division of National Informatics Centre, 
        \item With funding from DIT, MCIT
         \item Activities: 
              \begin{itemize}
                  \item Assistance in National Policy on Open Standards for e-Governance
                  \item Recommending and promoting the adoption of Open Source Software Stack
                  \item Provisioning support mechanism for Open Source Software Stack
               \end{itemize}
    \end{itemize}
\end{frame}
\begin {frame} {Sucess: Open Technology Centre}
   \begin{itemize}
         \item NIC mail server - 2,00,000+ mailboxes
          \item E-procurement system (tendering) across states 
          \item E-court project - 12000+ Linux laptops across India
          \item Common Integrated Police Application (CIPA)
          \item Trained around 100 teams on FLOSS at various government departments
    \end{itemize}
\end{frame}

\begin{frame} {AMTRON}
     \begin {itemize}
          \item Assam Electronics Development Corporation Ltd (AMTRON)
          \item Promotion of communication and information technology
           \item Technology hub for Assam and other north east state
      \end {itemize}
\end{frame}

\begin{frame}{Sucess: AMTRON}
  \begin {itemize}
      \item Anundoram Borooah Award Scheme 2005
          \begin {itemize}
             \item Free computers (powered by Linux) awarded to deserving students
              \item Almost 1,00,000 Linux PCs distributed so far 
              \item Loaded with with Ubuntu Operating System + educational packages (scilab, GNUPlot, Dr. Geo, step and Kalzium)
           \end{itemize}
       \item FOSS labs set up in 1610 secondary schools
       \item  900+ schools, students  are taught on the GNU/Linux platform
        \item 2,00,000+ students learn FOSS based computing every year
        \item More than 43.78 million INR (approx. 1 million USD) saved in five years
   \end{itemize}
\end{frame}
%modified work ends here

\end{document}

